\chapter{Examples}

This section contains examples of characters, combat sequences and other information that may clarify the system described in the previous pages. If any of the items in this system is ambiguous for the group, players can decide which way to interpret it.

\section{Character creation}

Here are some examples of character creation, following the rules laid out in the previous chapter. While the character names, races and available attribute points are the same, the examples show different approaches to creating a character for a given game. 

\subsection*{Brutus the Orc}

Here is Brutus, the orc. It has been created as a standard, generic character for a fantasy game. This is the kind of character you would be creating if your game is D\&D inspired or the group is looking for something easy and familiar.

\begin{labeling}{charcrre}
\item [Name] Brutus
\item [Race] Orc
\item [Job] Mercenary
\item [Virtue] Reputation
\item [Defect] Fear of water
\item [Attributes] (18, 5, 8, 6, 8, 9) \linebreak
STR 18 (+4)\linebreak
INT 6  (-2)\linebreak
CON 9  (--)\linebreak
DEX 8  (-1)\linebreak
CHA 8  (-1)\linebreak
\item [Skills] (1 + 3: 4 IP, 1 saved for later)\linebreak
Hand to hand combat +2 [STR]\linebreak
Intimidation +1 [STR/CHA]
\item [Equipment] Shield, sword, armor
\end{labeling}

\subsection*{Alternate character}

Here is Brutus, the orc. It has been created for a different kind of game. Brutus is a florist, likes animals and is afraid of water. This is the kind of character you would be creating if the group is looking for a calm adventure that involves more roleplaying and problem solving and less traditional fantasy combat situations.

\begin{labeling}{charcrre}
\item [Name] Brutus
\item [Race] Orc
\item [Job] Florist
\item [Virtue] Reputation
\item [Defect] Fear of water
\item [Attributes] (18, 5, 8, 6, 8, 9) \linebreak
STR 6 (-2)\linebreak
INT 8  (-1)\linebreak
CON 9  (--)\linebreak
DEX 8  (-1)\linebreak
CHA 18  (+4)\linebreak
\item [Skills] (1 + 3: 4 IP, 1 saved for later)\linebreak
Gardening +2 [INT]\linebreak
Befriend +1 [CHA]
\item [Equipment] Empty watering can, pruning shears, romance book
\end{labeling}

\subsection*{Nonsense brainstorm character}

Here is Brutus, the orc. It has been created for a game where the characters, settings and situations are created on the go and are based on whatever comes to mind first. These kind of games focus on pure improvisation and brainstorming, and are good when the group needs to take a break from the more traditional TTRPG experience and explore situations that may not show up in their regular games. Notice that, in this case, the skills, virtues and defects have no descriptions and have to be interpreted after creation.

\begin{labeling}{charcrre}
\item [Name] Brutus
\item [Race] Orc
\item [Job] Physicist
\item [Virtue] Cardboard fingers
\item [Defect] Cloud obsession
\item [Attributes] (18, 5, 8, 6, 8, 9) \linebreak
STR 18 +4)\linebreak
INT 9  (--)\linebreak
CON 8  (-1)\linebreak
DEX 8  (-1)\linebreak
CHA 6  (-2)\linebreak
\item [Skills] (1 + 1 + 1: 3 IP, 2 saved for later)\linebreak
Staring +1 [CHA]\linebreak
Lamp +1 [DEX]\linebreak
LET BRUTUS HELP +1 [CHA]
\item [Equipment] pocket lint, helmet, cute mug
\end{labeling}

\section{Roleplay and storyline progress}

Let's assume we are in a game with the second character, Brutus the Florist. They are on their flower shop and have to water the plants, but they are completely terrified of water. 
\par
One of the other PCs, Chonkers the Dwarf, enters Brutus' shop. In the following exchange, Brutus tries to convince Chonkers to help them, despite not knowing each other (Since this is their first interaction in the game)
\par

\begin{tabular}{rl}
Brutus& Welcome to the flower shop, can I help ya?\\
Chonkers& Yeah, I'm looking for the flower that goes *loud roar*\\
Brutus& Oh, the man-eating dragon plant of doom!\\
Chonkers& Do you have it?\\
Brutus& Yeah! Have a few in the storeroom, give me a minute\\
Brutus' player& I go to the storeroom to pick up the flower\\
GM& OK, it's there where you left it\\
Brutus' player& I return with the plant\\
Brutus& Here you have. It's a pretty rare plant, not very common to be asked for it\\
Chonkers& Well we have a man infestation, you see. Need something to thin them down somewhat\\
Brutus& Oh, a good choice then! Anything else?\\
Chonkers& Nothing else, thank you\\
Brutus& That'll be 200 gold coins, please\\
Chonkers& Woah there! That's pricey, ain't it?\\
Brutus& It is! These little marvels take quite a bit of effort to grow\\
Chonkers& Have any affiliate program I can sign up to? Not sure if this is within my budget\\
Brutus& Well, no, but I could give you a discount if you helped me water the plants\\
Chonkers& Hm. OK, fine. As long as you don't have any dwarf-eating dragon plants of doom.
\end{tabular}

\section{Combat}

This section contains a few examples of combat exchanges: Hand-to-hand, ranged combat and magic attacks. 

\subsection*{Hand to hand combat}

Chonkers the Dwarf, while watering some unknown plants in the giant storeroom, has disturbed one of the rats that lives there. The rat is visibly bothered and upset, and decides to attack Chonkers. 

\begin{tabular}{rl}
GM& The rat jumps at you before you can react! Roll DEX to dodge the attack.\\
Chonkers' player& *rolls a 12* That's 12, minus 1 for my DEX bonus\\
GM& *rolls 11* OK, that's 11 plus 2 of the rat's DEX bonus, so you get hit\\
Chonkers' player& Ouch. \\
GM& *rolls 10* That's 10 minus 3 of the rat's STR bonus. It bits you in the hand, but it's a \\
& minor scratch.\\
Chonkers' player& OK, now it's my turn! *rolls 15* 15-1, that's a 14\\
GM& *rolls 11* 11+2, that's 13. You hit the rat!\\
Chonkers' player& *rolls 15* 15, plus 2 STR bonus, that's 17!\\
GM& Well, you hit the rat pretty hard. It's wounded and decided to flee.\\
Chonkers& Darn rats...\\
\end{tabular}

\subsection*{Ranged attack}

Chonkers the Dwarf has made the rat flee, but the critter is still staring at them from the distance. The player decides to throw a rock at the rat to drive it away permanently.

\begin{tabular}{rl}
GM& The rat has fled, but it's now looking at you from the edge of the window.\\
Chonkers' player& I throw a rock at it!\\
GM& OK, roll dexterity.\\
Chonkers' player& *rolls 15* that's 15-1, 14. \\
GM& *rolls 5* 5+3 for the rat, 8. You hit it!\\
Chonkers' player& Yeah! Now for the damage... *rolls 10* 10+2, 12\\
GM& You hit the rat with the stone. The rat jumps and runs out the window, never to be \\ & seen again\\
\end{tabular}

\subsection*{Magic attack}

Once the rat situation has been taken care of, Chonkers the Dwarf continues watering the plants. One blue, vaguely threatening stem in a stone pot draws their attention. They walk closer and start watering the pot. Since Chonkers has no knowledge about plants, they water it too much, raising a small water elemental that bursts out of the pot.

\begin{tabular}{rl}
GM& The water elemental has splattered itself everywhere! There is a fairly large puddle in front of \\
& you, glowing threateningly. Suddenly, you see a flash, and what looks like a blue orb flying \\
& towards you! The elemental has attacked you with their magic!\\
Chonkers' player& Oh come on, I'm a dwarf! I don't even like water!\\
GM& *rolls12* The elemental's INT roll is 12.\\
Chonkers' player& *rolls 3* Well, that's bad. 3-1, 2. \\
GM& You receive the full force of the attack, which feels as unpleasant as having a shower. \\
& You start smelling good too, ew! By the time you can open your eyes again, the elemental \\
& has gone back to their pot.\\
Chonkers' player& OK, fine. Let's not do that again.\\

\end{tabular}