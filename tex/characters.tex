\chapter{Characters}

Fast character creation is one of the principles of SAO. As with other TTRPGs, there are no limits when it comes to characters. Players can make characters that represents themselves, create new fictional characters, or create characters based on any other real or fictional entity.
\par 
As a narrative focused system, character background is more important that combat skills, and it's recommended that players create their characters taking the storyline and their involvement in it in mind, rather than combat and the skills they will need for it.


\section{Character creation}

The following are needed to create a character:

\begin{enumerate}
\item Name
\item Race and Job
\item Attributes
\item Skills
\item Spells*
\item Defects and Virtues
\item Equipment
\end{enumerate}

Details for each item are provided in the following sections

\subsection{Name}

No rules here. Pick any name you'd like for your character.

\subsection{Races and Jobs}

There is no set list for races and jobs. The player can pick any combination they want unless, optionally and if the group decides so, it doesn't fit the general setting of the game.
\par
Races and jobs don't give any numeric bonus to the character, but can influence the storyline or give bonuses to certain actions or situations if the GM sees it fit. 
\par
If the group decides to have these bonuses, and to avoid inconsistencies in their application, they can be given in the form of attribute boosts when creating the character (e.g. elves can have dexterity bonuses)

\subsection{Attributes}

SAO uses 5 attributes: Strength (STR), constitution (CON), intelligence (INT), dexterity (DEX) and charisma (CHA).
\par
To deterime the character's attributes, roll 3D6 6 times and write down the totals for all the rolls. Choose 5 of those 6 numbers, and distribute them among the 5 attributes as you see fit.
\par 
Once the game has started, the cost of improving an attribute is 10 Nonsense Points (NP). An attribute be improved up to level 20.

\subsubsection*{Strength}
Strength measures the physical capacity of a character. The outcome of actions like moving heavy objects, attacking with blunt weapons or kicking a ball depends or this attribute.

\subsubsection*{Constitution}
Constitution is the physical endurance of a character. Running, resisting hits, holding your breath or drinking contests are some things that require high constitution.

\subsubsection*{Intelligence}
A character's intelligence shows their mental capacity. Both advanced or complex actions such as filing your taxes and throwing spells and simple things like speaking, writing or understanding how things work require a certain intelligence level or, failing that, access to the right tools (TurboTax licenses are not included with this manual)

\subsubsection*{Dexterity}
Dexterity is a character's capacity to jump, juggle and maintain equilibrium with ease. This attribute is used for both ranged weapon use and actions that require this balance and coordination (Climbing, playing DDR or sneaking)

\subsubsection*{Charisma}
Charisma relates to how likable a character is. A character with high charisma will be able to befriend, lie, deceive and convince easily. On the other hand, a character with low charisma will have trouble starting conversations with strangers. 

\subsection{Skills}

As with races and jobs, SAO has no set list of skills to choose from.
\par 
The player, initially, has 5NP to distribute among skills they see fit, up to 4 skills. The first level costs 1NP, the second 3, the third 5, and so on ($P_{n}=\lceil1.5n\rceil$). Once the game has started, the cost of adding a new level 1 skill is 5NP. Points not spent during character creation can be stored for later use. 
\par
All characters start with the level 1 Nenify skill, wether they like it or not (See Mechanics $\rightarrow$ Nenify for further details)

\subsection{Spells}

If the race, job or background of a character is magical in nature, the player has 1NP to spend in a magical skill (Spells, for example). As with regular skills, there is no set list, so it's up to the player's imagination to define spell names, descriptions, effects, costs and so on.
\par 
This NP is lost if not used during character creation, and it can only be used for magical skills. The cost of upgrading a magical skills is $Pm_{n}=2n$ (4 for level 2, 6 for level 3 and so on). The cost of adding a new magical skill during the game is 6NP.

\subsection{Virtues and Defects}

Once the attributes, race, job and skills of a character are set, an outline of the kind of being it is will start to emerge. However, even with the same name, race and skills, every character in the world is unique and has its own advantagas and disadvantages that come with their storyline and background.

At this point the player can choose, optionally, to add up to 3 virtues and 3 defects to their character. As usual there is no set list and the specifics are up to the player.

A virtue is an innate ability of the character, and a defect is a characteristic that makes their life slightly more difficult. Examples of virtues are "Good with animals", "Good athlete", "Kind person" or "Quick learner". Examples of defects are "Bad with math", "Clumsy", "Too Aggressive" or "Unable to explain".

\subsection{Equipment}

Once the character is created, the player adds to their inventory the basic equipment that character would use or need on a regular day of their life (Weapons and armor, tools, books or notes, clothing, etc). 
\par
This list doesn't need to be exhaustive, and its purpose is narrative rather than to have a strict control of what the player is carrying, so there is no need to add a hilariously large amount of weapons (Unless the character would normally carry them) or a specific number of gold coins (Or even coins, if it can be assumed that the character would carry money with them).
\par 
This can be skipped if the setting requires so (e.g. if the characters start their adventure in a jail cell)

\subsection{Others}

Unless a defect or a story situation prevents them to, characters speak at least one language. In a normal game all characters share at least one language and there is no need to worry about inter-PC communication during the game. However, bear in mind that this is another element of your character the group can use to develop the setting and the story. If this is the case, remember that language related skills can be added like the rest of skills.
\par 
If the players are familiar with TTRPG, the characters have been in other adventures before or the characters have been adapted from campaigns in other systems, additional details such as reputation, contacts and extra information can be included in the character sheet. This information can be relevant to the story and can be used as an additional element in the plot.
\par 
As a final note, remember that, regardless of the basics laid out here, you have total freedom to create any character you wish. Feel free to modify or ignore any of the previous rules and steps if it helps you get closer to the character you want to play. 
\par
Remember that the important thing is not the attributes, skills or numbers of your character, but to feel comfortable in their skin and to have fun while roleplaying that character with your friends.

\subsection{Character review}

Once all players have created their characters, They should present themselves to the rest of the group (If the PCs know each other in the game setting) and show their character sheets to the GM to make sure everything is balanced and there is nothing that could seriously impact everyone's experience.