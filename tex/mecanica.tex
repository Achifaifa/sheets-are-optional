\chapter{Mecánica}

La mecánica de Tres Reinos está centrada en la narración, y usa el combate como algo puntual para añadir elementos a la historia. 

\section{Mecánica de Juego}

La mecánica de juego es igual que la del resto de juegos de rol: El master expone la situación de los personajes a los jugadores, los jugadores deciden qué hacen sus personajes, y el master les expone la consecuencia de sus acciones y la nueva situación. 

Las situaciones y aventuras se pueden sacar de alguna plantilla o situación real o ficticia, se pueden preparar desde cero con antelación o se pueden improvisar sobre la marcha, todo depende de qué tipo de experiencia prefieran los jugadores. 

Si los jugadores deciden hacer que sus personajes hagan algo que pueda suponerles dificultad (Ganzuar una cerradura, descifrar un pergamino o darse de baja en la compañía telefónica), deberán tirar un dado para determinar el resultado de su acción, aplicando a la tirada los bonuses que tenga su personaje (Por atributos y por habilidades) y restándole una penalización por dificultad o factores externos si fuera necesario. Los resultados pueden ir desde un fallo total (1) hasta un éxito apabullante (20). Tras una tirada, el master de la partida se encargará de describir las consecuencias de la acción (Más o menos gráficamente) y de exponer la nueva situación. 

Como el objetivo del juego es jugar a rol y no pasarse un día entero tirando dados, para acciones sencillas como ir de una ciudad a otra o sentarse en una silla no es necesario realizar una tirada, aunque si los jugadores están de buen humor pueden elegir tirar un dado ``por las risas'' al anunciar su acción para ver si el resultado es especialmente bueno o malo. Generalmente esto se suele hacer al jugar con la regla de ``Pifias supercríticas'' para echarse unas risas, o cuando el jugador necesita que esa acción salga especialmente bien si cree que puede afectar a la trama (Ver el apéndice B (Variantes) para más información). Puede ser gracioso puntualmente, pero se recomienda no abusar.

\section{Mecánica de Combate}

Una vez que los personajes empiezan a combatir, comienza un nuevo encuentro. El encuentro comienza cuando los jugadores atacan a un monstruo o NPC o cuando un monstruo o NPC les ataca o es alertado de la presencia de los jugadores en situaciones que impliquen sigilo o engaños. El encuentro acaba cuando los jugadores huyen o están fuera de peligro o cuando uno de los dos bandos muere o es dejado fuera de combate. 

Al principio de cada encuentro se decide el orden en el que los combatientes se moverán (Puede decidirse en base a la situación o bien echarlo a suertes con una tirada de iniciativa (D20 + bonus de destreza). Después, los combatientes actúan siguiendo ese orden hasta que el encuentro acaba.

Al iniciar un encuentro, especialmente si es un encuentro con varios enemigos, se recomienda utilizar algún tipo de token para representar dónde está cada jugador y cada enemigo. No influye en el juego, pero ayuda a visualizar las situaciones y ser coherente al determinar el resultado de las acciones y los ataques.

Un jugador puede realizar cualquier acción en su turno: Moverse, atacar, lanzar un hechizo, o sentarse a leer el periódico. A continuación se explican las más comunes:

\subsection{Ataques físicos}

Los ataques físicos son los ataques lanzados cuerpo a cuerpo. Los ataques con armas de mano (Hachas, espadas, pan duro...) tienen una mecánica sencilla. 

\begin{enumerate}
\item Se compara una tirada del atacante (Fuerza o agilidad) con una del defensor (Agilidad)
\item Si el resultado del atacante es mayor, el ataque tiene éxito
\item Si el resultado del defensor es mayor, el ataque es bloqueado. Si la diferencia es 15 o mayor el defensor puede contraatacar.
\end{enumerate}

\par 
El grado de éxito del ataque depende de la diferencia entre las tiradas. 
\par 
Una vez que un ataque tiene éxito, el atacante realiza una tirada de fuerza para determinar el daño realizado. 

\subsection{Ataques a distancia}

Los ataques a distancia funcionan de manera similar a los ataques físicos, excepto que el atacante realiza una tirada de agilidad para atacar. El defensor realiza una tirada para esquivar el disparo, pero no puede optar a contraataque. 

\subsection{Ataques mágicos}

La mecánica de los ataques mágicos es parecida a la de los ataques a distancia. 

\begin{enumerate}
\item El atacante realiza una tirada de inteligencia
\item Dependiendo del hechizo usado, el defensor podrá optar a una tirada de agilidad para esquivar el ataque.
\item Si el resultado del atacante es mayor de 10 (El master puede decidir aplicar penalizaciones en caso de que la situación o el hechizo sean especialmente difíciles) el ataque tiene éxito. 
\item Si el defensor puede optar a la tirada para esquivar el ataque y su resultado es mayor, el ataque es esquivado, pero no podrá contraatacar si la diferencia es mayor de 15 a no ser que el atacante se encuentre dentro de su rango de ataque
\end{enumerate}

\par 
El daño de un ataque mágico depende del hechizo en sí y no de una tirada de dados, pero se puede utilizar el resultado de la tirada de inteligencia como indicador de la potencia del hechizo o especificar una tirada de dados para ese hechizo al crear el personaje.

\subsubsection*{Heridas}

El daño realizado no se mide en puntos de vida, sino que se tiene en cuenta en heridas. Un personaje con heridas superficiales puede seguir combatiendo. Una herida grave tiene una penalización de -1 a futuras tiradas, y dos heridas graves tienen una penalización de -3. A la tercera herida grave, el personaje muere o pierde el conocimiento y deja el combate.

\section{Frikipuntos de experiencia}

El master puede dar frikipuntos de experiencia a los jugadores, que pueden usarlos para mejorar sus personajes. Estos puntos pueden ser otorgados por matar monstruos, salir de situaciones difíciles, buenas interpretaciones, soluciones ingeniosas a problemas... No hay límite establecido, pero se recomienda no excederse y ser coherente con la situación en la partida y los costes de añadir-mejorar habilidades.

\section{Nenificar}

Nenificar es una habilidad innata de cualquier habitante del continente. Al nenificar, el personaje cuenta un chiste malo (1D20 + nivel de nenificar). 
\par 
Si en algún momento de la partida un jugador cuenta un chiste malo, los jugadores y el master pueden decidir añadir un nivel de nenificar al personaje de ese jugador. Esta habilidad puede usarse para que el personaje cuente chistes malos en la partida, causando diversos efectos según la situación.
\par 
Cuando un personaje nenifica a otro, este sufre los efectos del chiste malo (Desde una risita leve a un aneurisma). La única manera de salvarse de los efectos es intentar devolver la nenificación contando un chiste peor (Tirada de dados mayor).
\par 
Las nenificaciones no letales son habituales en el País Juasco y Jarlicia, pero en el reino de Nofarra una mención leve puede llamar poderosamente la atención de los ciudadanos y la guardia real, y las consecuencias pueden ser negativas e inesperadas. Se recomienda no abusar de este poder.