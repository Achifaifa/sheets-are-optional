\chapter{Mechanics}

As a narrative-focused system, Sheets Are Optional uses combat as another story element, and die for the sole purpose of preventing the GM to have total control over the events and their outcome. 

\section{Game mechanics}

The overall mechanic is identical to that of most TTRPG systems. The Game Master (GM) describes a situation the Player Characters (PCs) are in. The players decide what their characters would do in that situation, and then the GM describes the consequences of those actions and the new situation the PCs are in.

The situations, adventures and events of the story can be taken from a pre-made module, a real or fictional situation or can be improvised, either right before the game or as the events unfold. It all depends on the kind of game the players want and are comfortable with.

If the players decide their characters will do something that may be difficult or have unexpected consequences (Picking a lock, petting a feral dog or transferring their phone number to another company), a dice is rolled to determine the outcome of that action, applying a bonus (If the character is good at it or the situation is favourable to them) or a penalty (If the character is inexperienced or the situation is not favourable) to that roll. The results of these rolls will vary from a total disaster (1) to an outstanding success (20). After a roll, the GM will describe (More or less graphically) the outcome and consequences of the action, and will describe the new situation derived from that action.

Since the goal of this system is to roleplay and not to spend half the session rolling die, no rolls are necessary for simple actions with obvious outcomes, such as walking around or sitting on a chair, although the players may agree to roll that 'for laughs' if they feel like it when announcing their action. This may be done for fun when playing with variants such as 'supercritical fumbles', or when the player wants the outcome of a simple action to be particularly good for plot reasons. In any case, abusing dice rolls is discouraged, since it slows down the action and distracts players from the narrative.

\section{Combat mechanics}

If, for whatever unexpected reason a fight breaks out, a new encounter is started. An encounter starts when players attack first or when an Non-Player Character (NPC) or monster attacks them or is alerted of their presence in certain situations (Or whenever a fight would break out for narrative reasons). The encounter ends when the PCs flee or are out of danger, when all but one of the sides are incapable of fighting or when the involved sides agree to stop fighting.

At the start of an encounter, the order in which everyone will act is decided. This can be done either narratively (Based on the situation or PC motivation) or randomly with a dice roll (1D20 + DEX bonus of the PC. See the appendices for bonus tables). Once this is decided, players make their actions in this order until the encounter ends.

Once an encounter is started, specially if it's an encounter with several people or groups involved, the use of tokens over a board or map representing everyone involved is encouraged (They don't have to be perfectly painted minis, a few beans marked with a pen work too). It has no influence on the game and it's not part of the base mechanics, but it helps visualizing the situation and being coherent when determining the outcome of actions and attacks.

A PC can do any action in their turn. Move, attack, launch a spell, read the newspaper or call their mom for advice. Anything, really. Combat related actions and their mechanics are described next.

\subsection{Physical attacks}

Physical attacks are those done with either bare hands or handheld weapons (Axes, swords, phone directories or stale bread), and they have a fairly simple mechanic.

\begin{enumerate}
\item Compare the attacker's dice roll (STR or DEX, depending on the attack) with the defender's (DEX)
\item If the attacker's is higher, the attack succeeds.
\item If the defender's is higher, the attack is blocked. If the difference is 15 or higher, the defender can counterattack (Repeat this process, but with the attacker and defender roles swapped)
\end{enumerate}

\par 
The degree of success of the attack depends on the difference between the attacker's and defender's rolls.
\par 
Once an attack succeeds, the attacker makes a STR roll to determine the damage of the attack. 

\subsection{Ranged attacks}

Ranged attacks work in a similar fashion to physical attacks, except the attacker has to make a DEX roll. The defender also rolls DEX to dodge the attack, but can't opt to a counterattack.

\subsection{Magic attacks}

Magic attacks work like ranged attacks, with some exceptions and changes depending on the spell used. 

\begin{enumerate}
\item The attacker makes an INT roll
\item If the attacker's roll is greater than 10 (The GM may decide to apply bonuses or penalties to the roll, or the spell may have different requirements), the attack succeeds
\item The defender may opt to a dodge roll (Using DEX) depending on the spell.
\item The defender dodges the attack if their DEX roll is higher, but may only opt to a counterattack if the spell used is casted from a short range or the spell or situation allow it
\end{enumerate}

\par 
The damage done depends on the spell itself and not a dice roll. If needed, the INT roll can be used to determine spell strength, or an additional roll can be specified when adding the spell to the character

\subsubsection*{Wounds}

Damage done is measured in wounds, rather than Hit Points. A PC with superficial wounds can go on as usual. A serious wound has a -1 penalty to future rolls until healed, and two serious wounds have a -3 penalty. When a PC is inflicted with a third serious wound, it dies or losses consciousness and leaves the fight. 

\section{Nonsense Points}

The GM can award Nonsense Points (NP) to players, which can use them to improve their characters. These points can be awarded for defeating enemies, getting out of difficult situations, good roleplaying, ingenious solutions, etc. 

There is no set limit, but it's recommended to familiarise oneself with the system and the NP cost of things before wildly distributing them.

\section{Nenify}

If, at any point in the game, a player (Not the PCs, but the players themselves) tells a bad joke or makes a funny comment, the group can decide to increase the PCs Nenify level by 1. This skill can be used to make the PC tell bad jokes in game, causing various effects depending on the situation.
\par
When Nenifying someone, the PC tells a bad joke (1D20+Nenify bonus)
\par 
When a PC Nenifies someone, the targets suffers the effect of the bad joke (Anything from a mild smile to uncontrollable laughter to an aneurysm if the joke is bad enough). The only way to save oneself from these effects is to return the Nenification by telling a worse joke (Higher dice roll)