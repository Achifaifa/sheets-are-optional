\chapter{Appendix B: Variants}

Once players and GMs are familiar with TTRPGs and is familiar with the most common game mechanics and stories, they may need something different from time to time to keep things fresh. 
\par
This appendix contains some rule changes and variants (Non necessarily exclusive to the SAO system that can add some spice to the game. The use of these variants is optional.

\section{General mechanic variants}

\subsection*{Diceless}

Sometimes there are no dice around, or the group wants to focus on roleplaying rather than rolling die and crunching numbers. Dice rolls can be avoided altogether by comparing the character's skill levels and attribute bonuses directly, or having a pass/fail threshold for actions that require a certain skill level to perform. While this makes any action predictable and removes the unpredictability die provide (Which may force players to think twice before performing a dangerous action), it greatly increases the game speed, allowing the group to 'get over' combat sequences and go back to roleplaying faster.
\par 
If this version is too extreme, the consequences of an action can be completely left to the GM, which can add an element of randomness by privately throwing a single dice or flipping a coin and communicating the outcome to the rest of the group as if that dice had never been rolled.
\par 
While it's important that the players have a certain level of trust with the GM in any game, it becomes specially important when using this last method. Make sure everyone in the game is OK with this variant before implementing it in your game.

\subsection*{Ultracritical rolls}

Whenever a fumble (1 on any roll) is rolled, the result is absurdly disastrous in the most ludicrous, over the top way imaginable. This can be compensated with the opposite for good rolls (20)

\subsection*{Open rolls}

When a player rolls 1 or 20, it's possible to repeat the roll to save oneself from the fumble (when rolling 1) or improve the result further (when rolling 20). Any good or bad results obtained in the first roll can be further improved or worsened by this second roll.

\subsection*{Nobody's perfect}

When choosing the attribute values from the dice rolls to their character, the player has to add the lowest roll to their selection. 
\par
e.g.: If the results are 14, 5, 10, 8, 12 and 10, the player has to use the 5 on one of the attributes

\section{Combat}

\subsection*{Carlos Duty}

Players can recover from serious wounds by avoiding damage during combat for 3 turns

\section{Environment and Others}

\subsection*{Replay}

At any moment during the game and once per day, PCs can use an amulet that returns them 5 minutes into the past, activating a roulette that chooses an unfavourable event randomly from a list.
\par
It's recommended that the full list is created and only visible by the GM.

\subsection*{Me no parlo spraje}

Different languages inside the game have to be represented with different languages between players. No common languages are allowed unless present in game.

\subsection*{Sssshh!! We're roleplaying!}

Everything players say during the game is considered spoken by their characters, regardless of the sentence and the situation.

\subsection*{Taboo}

Before a game, players have to agree on at least one taboo word. When a players says any of those words and the GM notices it, their next roll is an automatic 1.