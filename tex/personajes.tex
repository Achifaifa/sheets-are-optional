\chapter{Personajes}

Los personajes son muy fáciles de crear en Tres Reinos. No hay límites en cuanto a creación de personajes (Un jugador puede hacer un personaje que sea él mismo, o bien crear un personaje ficticio completamente diferente). Los combates son algo puntual en la versión estándar del juego, así que no es imprescindible que un personaje tenga una capacidad de combate espectacular para avanzar en la historia. 

\section{Creación de personajes}

Para crear un personaje necesitas los siguientes elementos:

\begin{enumerate}
\item Nombre
\item Raza y profesión
\item Atributos
\item Habilidades
\item Hechizos*
\item Defectos y virtudes
\item Equipo
\end{enumerate}

En las siguientes secciones están los detalles de cada uno de los elementos

\subsection{Razas y profesiones}

Este juego no tiene una lista predefinida de razas y profesiones. El jugador puede elegir cualquier combinación que sea coherente con la ambientación de la partida. 
\par
La raza y la profesión no tienen por qué dar ningún bonus numérico al personaje, pero pueden influir en el transcurso de la historia o dar alguna bonificación a ciertas acciones si el master lo ve conveniente. En caso de que sea necesario, y para evitar inconsistencias en las bonificaciones aplicadas, se pueden aplicar esas bonificaciones a los atributos al empezar la partida. Por ejemplo, los elfos pueden tener bonificaciones a agilidad, los enanos a fuerza, etc.

\subsection{Atributos}

Tres Reinos utiliza 5 atributos: Fuerza, Constitución, Inteligencia, Agilidad y Carisma. 
\par
Para elegir los atributos tira 3d6 6 veces y anota los resultados. Elige los cinco números que quieras (Normalmente los más altos) y asígnalos a los atributos de tu personaje que quieras.
\par 
No es imprescindible, pero sí recomendable que los atributos elegidos sean coherentes con la raza y la profesión del personaje.
\par 
Una vez dentro de la partida, el coste de mejorar un atributo es de 10 frikipuntos de experiencia. Un atributo sólo puede ser mejorado hasta llegar al nivel 20.

\subsubsection*{Fuerza}
La fuerza indica la fuerza física de un personaje. Acciones como mover objetos pesados, atacar con armas cuerpo a cuerpo y sujetar a otro personaje dependen de este atributo. 

\subsubsection*{Constitución}
La constitución es la resistencia física de un personaje. Correr, resistir golpes, aguantar la respiración o participar en concursos de bebedores son acciones que requieren una constitución alta. 

\subsubsection*{Inteligencia}
La inteligencia de un personaje indica su capacidad mental. Tanto acciones avanzadas como lanzar hechizos y descifrar lenguajes como acciones simples (Hablar, escribir, entender mecanismos) necesitan cierto nivel de inteligencia.

\subsubsection*{Agilidad}
La agilidad es la capacidad de un personaje de saltar, realizar acrobacias y mantenerse en equilibrio con facilidad. Este atributo se utiliza tanto para utilizar armas a distancia (Arcos, ballestas, etc) como para realizar acciones que requieran de destreza (Escalar, jugar al DDR, andar sigilosamente...).

\subsubsection*{Carisma}
El carisma es la habilidad de un personaje de, en general, causar una buena impresión en los demás. Un personaje con un carisma alto podrá mentir, engañar y convencer con mayor facilidad. Por otro lado, un personaje con un carisma bajo tendrá problemas para entablar conversaciones sencillas con un desconocido.

\subsection{Habilidades}

Al igual que con las razas y las profesiones, este juego no tiene una lista predefinida de habilidades entre las que escoger. 
\par
El jugador tiene 5 frikipuntos a repartir entre las habilidades que él o ella crea conveniente, hasta un máximo de 4 habilidades. El primer nivel cuesta un punto, el segundo 3, el tercero 5, y así sucesivamente ($P_{n}=2n-1$). Una vez empezada la partida, el coste de añadir una nueva habilidad es de 5 frikipuntos. Los puntos que no se usen durante la fase de creación de personaje podrán guardarse para más tarde.
\par
Cada nivel de habilidad añade un +1 a las tiradas relacionadas con esa habilidad.
\par
A las habilidades elegidas se les añade un nivel de "Nenificar"


\subsection{Hechizos}

Si la raza o la profesión de un personaje están relacionados con la magia, el jugador dispone de un frikipunto con el que puede añadirse una habilidad mágica (Conjuros, hechizos, etc). Al igual que con todo lo demás, no hay una lista de habiliades predefinida y todo queda a la imaginación del jugador.
\par
Este frikipunto se pierde si no se usa, y sólo puede usarse para habilidades mágicas. El coste de subir una habilidad mágica es de $Pm_{n}=2n$ (4 para el segundo nivel, 6 para el tercero, etc). El coste de añadir una habilidad mágica durante la partida es de 6 frikipuntos. 
\par 
El master y el jugador deberán decidir qué habilidades son razonables y bajo qué condiciones se pueden usar (Pista: "Destruir universo" no suele una habilidad mágica razonable, aunque el master siempre puede dar su visto bueno y utilizarlo como un elemento de la historia)

\subsection{Defectos y virtudes}

Una vez decididas las habilidades y el tipo de personaje, un jugador puede escoger añadir hasta 3 defectos y 3 virtudes a su personaje. Tampoco hay lista predefinida, el master deberá ver la lista (Y el jugador usar su sentido común) para que estos defectos y virtudes estén balanceados y tengan sentido.

\subsection{Equipo}

Con el personaje finaliado, cada jugador añade a su inventario el equipo básico de su raza y profesión (Armas, armaduras, herramientas, etc). Normalmente conseguir equipo a lo largo de la partida no suele ser un problema, así que no hay necesidad de armarse hasta los dientes. El master puede elegir omitir este paso si la aventura o la ambientación lo requieren (e.g. si los personajes empiezan su aventura en un calabozo).

\subsection{Otros}

Los personajes creados hablan al menos un idioma (Excepto cuando un defecto se lo impide). Normalmente todos los personajes comparten un lenguaje común y no hay que preocuparse por la comunicación en la partida. Sin embargo, si un jugador elige un personaje de una raza extraña, de otro continente o con un defecto o característica particular, es posible que no comparta ese idioma común. Los demás jugadores pueden especificar los lenguajes hablados (O las habilidades relacionadas con el lenguaje) e incluir la barrera del lenguaje en la mecánica o en la historia.
\par 
En caso de que los personajes hayan jugado en otras partidas o juegos (Por ejemplo, personajes que los jugadores hayan manejado en D\&D adaptados a tres reinos para seguir la aventura con otra mecánica), de personajes usados en partidas anteriores o de personajes nuevos a los que el jugador quiera darles su toque personal, se pueden incluir detalles extra. Datos como la reputación, los contactos o los recursos disponibles, o información extra sobre el estado del personaje que no se haya incluido como defecto o virtud pueden apuntarse en la parte de atrás de la hoja de personaje. Esta información puede ser relevante para el transcurso de la historia y puede ser usada como un elemento más en la trama.

