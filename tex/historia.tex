\chapter{Historia}

Las relaciones entre los reinos del País Juasco, Nofarra y Jarlicia no han sido siempre tensas. Al menos no desde el inicio de los tiempos ni desde el año 0, cuando el humor era considerado el bien más preciado de la humanidad. 

\section*{Colonización inicial (0-39)}

Los primeros colonos llegaron al continente en el año 0, iniciando el calendario actual (CA). No sería correcto hablar de reyes y reinos en esta etapa de la colonización, ya que en los primeros momentos la presencia humana en el continente se limitaba a unos pocos asentamientos repartidos por el Golfo de Bizquera. El único movimiento de personas y mercancías entre estos asentamientos se realizaba por barco. El transporte por tierra, salvo para distancias cortas, era negligible, y la red de comunicaciones estaba limitada a los mensajes que los viajeros podían llevar consigo.
\par 
Sin embargo el estado general de las colonias era bueno, las tierras eran fértiles y las tierras sin explorar parecían prometedoras. Bajo estas circunstancias Arturius, comandante de la expedición al continente, decidió fundar un conglomerado de colonias independientes para facilitar el comercio y el transporte. Según la historia tradicional (Transmitida de manera oral en esta época y recopilada por los escribas de Pancho el Sabio al rededor del año 275), en poco tiempo los colonos consideraban a esta agrupación política y económica de asentamientos como un reino. Cinco años después, en el año 33, Arturios formalizó el puesto, unificó los asentamientos en el reino Arturiano y se proclamó rey.
\par 
Debido a las condiciones poco higiénicas de los asentamientos por aquellos tiempos, Arturios murió poco después en la que fue la primera casa erigida en el continente. Sin embargo su legado permanece, ya que todos los elementos sociales modernos (Sistema jurídico, Guardia Real, Cámara de Consejeros, etc) fueron creados durante su reinado en un frenesí de reformas orientadas a mejorar la vida de los colonos y facilitar la expansión y el comercio.

\section*{Expansión continental y fundación de los reinos (40-290)}

A la muerte de Arturios en el año 39, sus tres hijos se repartieron el reino como dictaba una de las tradiciones recientemente implantadas por el difunto rey. En la declaración de Ilbod (40 CA), realizada en acto solemne en la ciudad del mismo nombre (Por aquel entonces la colonia más grande) reconocía a los tres descendientes como reyes y daba, por fin, nombres a los tres reinos: País Juasco al oeste, Reino de la Farra en el centro y Jarlicia al este del continente. Las fronteras se marcaron debidamente para evitar confusiones, la guardia real se dividió por zonas y cada rey creó zonas administrativas basadas en las pocas colonias que había en su reino. 
\par 
Las fronteras, además de proporcionar una clara demarcación política, eran tratadas como algo místico por los habitantes del reino. Los soldados (Milicianos voluntarios en estas fechas, los ejércitos regulares vendrían más tarde) no se atrevían a cruzarlas estando de servicio. Era algo que, simplemente, no estaba justificado, ni siquiera al perseguir a criminales que, por supuesto, intentaban huir y escapar a reinos colindantes. Sin embargo había oportunidades de expansión en tierras no exploradas, y las llamadas de aventura, fortuna y gloria llamaban poderosamente la atención de habitantes para los que ser un colono había perdido el aura de aventura.
\par 
Bajo estas condiciones la colonización del continente avanzó a grandes pasos, y áreas que antes eran consideradas remotas y hostiles daban todo tipo de tesoros a quien se atrevían a explorarlas. A partir del año 47 Una red continental de caminos se construyó rápidamente entre los principales asentamientos, que empezaban a parecer ciudades primitivas, y se fue extendiendo a lo largo de los años hasta cubrir gran parte del mundo conocido por los colonos. 
\par 
Los avances tecnológicos, como las linternas de grasa de troll (c. 75) y las nuevas técnicas de preservación de comida (c. 90), mejoraban la vida de los habitantes en las zonas más pobladas, permitían a la gente dedicarse a otros oficios alimentándose del surplus y animaban a los exploradores a ir a lugares aún más remotos. En el año 151 la última zona sin explorar, el Cabo Norte, presenciaba la llegada de colonos en busca de yacimientos minerales. En 180 los dos exploradores supervivientes de la expedición en la que servían atravesaban el Bosque negro, la mayores extensión forestal del continente (En la actualidad permanecen las únicas personas que cruzaron el Bosque Negro con vida). En el año 207 el explorador Aver, desdendiente de una larga dinastía de aventureros y mercenarios, desentrañaba el misterio de la tumba solitaria (Una curiosa formación rocosa en el desierto que resultó ser fruto de la erosión, y no de habitantes anteriores del continente), y en el año 270 el célebre cartógrafo Apam, al servicio del rey Pancho el Sabio, concluyó su ``Atlas del Continente'', con mapas a gran escala que incluían datos incluso de los rincones más remotos del mundo conocido.
\par 
Con todo el continente explorado, los tres reyes podían centrarse ahora en explorar al máximo sus territorios y sacar el máximo provecho a sus recursos. Los reyes estaban satisfechos con los avances, pero una parte de la sociedad, principalmente los exploradores y aventureros que habían protagonizado la rápida expansión en el último siglo, pensaba que ya no había nada más que hacer en el continente. A no ser, por supuesto, que un reino pudiera, de alguna manera, utilizar o anexar las tierras de otro.

\section*{Conflictos iniciales (290-723)}

Entre la alta sociedad y los gobernantes del continente, la sensación de ahogo por la falta de nuevas tierras se hacía notar cada vez más. Durante los próximos siglos, generación tras generación, la avaricia de los gobernantes aumentaría también sin medida, alimentada por los tesoros descubiertos por los exploradores en la época dorada de la exploración continental y el gran volumen de comercio entre los reinos que, a pesar de todos los esfuerzos por parte de las partes implicadas, decrecían año tras año sin explicación aparente. 
\par
Los antepasados, fuente de orgullo, consejo y sabiduría en tiempos anteriores, se olvidaban cada vez con más frecuencia, las leyes Arturianas se ignoraban y los habitantes más ricos intentaban hacerse con tierras en territorio extranjero para ganarse el favor del rey. Hacia el año 506, los soldados se adentraban con cierta regularidad en otros reinos, en ocasiones robando y saqueando ``un poquito que casi ni se nota'', y en otras ocasiones quemando pueblos enteros en los que los criminales que estaban persiguiendo intentaban refugiarse. Según las fuentes escritas que permanecen, esto no sentaba bien a los otros reyes, y aún menos a la población. La situación empeoraba, los ataques se hacían más violentos y los tres reinos estaban más divididos que nunca.
\par
En el Reino de la Farra (Más concretamente en la ciudad de Riosas) y tras soportar saqueos, incendios y asesinatos varios durante varios meses consecutivos, un grupo de ciudadanos decidió formar un grupo de vigilantes en el año 720 y atacar directamente al País Juasco, hacia el que habían acumulado un odio irracional a lo largo de cientos de años. 
\par 
Sin embargo, a medida que avanzaban erráticamente por el reino reclutando soldados para su causa (Principalmente gente de escasa aptitud con horcas y antorchas), el rumbo de la masa se fue transformando. Hasta que un fatídico verano, tras clavarse una de las horcas en el pie y aguantar risitas durante dos semanas, uno de los líderes de la marcha anunció que ``No tenía ni puñetera gracia'' y pidió a todos que dejaran de reírse. Fue un mensaje tan absurdo, inesperado y chocante para los habitantes del reino que la frase ``Dejad de reíros coño'' corrió como la pólvora y, una vez perdido el contexto, empezó a causar revoluciones sinrisistas en prácticamente todas las ciudades. El fenómeno no tardó en expandirse primero a zonas fronterizas, y poco después al resto del continente. En el País Juasco y Jarlicia estas revoluciones fueron sofocadas y acabaron siendo una secta minoritaria, pero en el reino de Nofarra los rebeldes consiguieron invadir el palacio real disfrazados de sinrisistas durante una actuación teatral y mataron a la familia real, consiguiendo el poder (Que mantendrían hasta la actualidad). 

\section*{Guerra Continental (736-745)}
 
Tras las revoluciones del año 723, el reino de Nofarra adoptó una política muy estricta de seriedad y disciplina, separando y aislando a los demás reinos gracias a su posición privilegiada. Los puestos fronterizos se fortificaron, se implantó el sistema de control de aduanas que se sigue utilizando en la actualidad, se reunió a los mejores humoristas del reino para mandarlos a prisión y, en cuanto parte del reino estuvo bajo control, se declaró la guerra al País Juasco. 
\par
Una docena de mensajeros salieron en barcos de la corte de Ladislao (700-759), rey del país juasco, en dirección a varias de las mayores ciudades del reino Jasllego. Varias semanas después, el rey, estacionado en una ciudad costera del norte del reino, recibió el mensaje, aceptó la alianza propuesta por Ladislao y, creyendo que con la alianza la victoria se podía dar por sentada, renombró la ciudad en la que se encontraba tras el mensajero del que había recibido la noticia (La ciudad, Sto Chiquito de la Calzada, conserva su nombre en la actualidad).
\par 
De acuerdo con el plan provisto por Ladislao, las tropas de la alianza comenzaron a acumularse en las fronteras. Según el historiador Asipa, que escribiría las crónicas de la guerra 100 años después, ``Había tantos soldados en los prados fronterizos que eran confundidos con un bosque más. Todos los animales de los bosques cercanos mayores que una rata desaparecieron para alimentar a las tropas, y por donde el inmenso ejército pasaba, la hierba no volvió a verse en al menos dos generaciones''. 
\par 
El reino de Nofarra, a pesar de tener una capacidad mayor en cuanto a tecnología y soldados, necesitaba de mucho más tiempo para reunir sus recursos. Un viaje a pie de lado a lado del reino podía llevar meses, y mover tropas a través de la zona central, a través de cordilleras y rodeando el bosque negro, era complicado y llegaría a causar, en las últimas etapas de la guerra, más bajas que el ejército enemigo. La alianza de los dos reinos podría aprovechar esto para invadir nofarra y ganar cierta ventaja, pero los mensajes entre ambos reyes podían llegar a tardar varias semanas, y toda la operación dependía de una sincronización (Y una suerte) nunca vista hasta la fecha.
\par 
Sin embargo, no hizo falta que ninguno de los reyes diera la orden. En diciembre del 736 un pequeño grupo de tropas Jasllegas, afectadas por la malnutrición y la escasez de recursos en su lado de la frontera, decidió entrar en territorio Nofarro cruzando un bosque poco vigilado y saquear Detal, una pequeña ciudad sin fortificar que había acumulado cierta riqueza por su posición cerca del paso fronterizo con el mismo nombre. Al amanecer, nos cuenta Asipa, ``La ciudad, sencillamente, había dejado de existir. Donde antes había edificios, iglesias, comercios y personas, ahora había escombros. Sólo escomros, y nada más que escombros''. Según un informe escrito por los exploradores Nofarros enviados al paso fronterizo, ``No se ha encontrado Detal, a no ser que el montón de piedras al lado del bosque sean la ciudad que buscábamos. No encontramos ningún rastro de actividad humana. Ni cadáveres, ni herramientas, ni restos de ningún tipo. Solo un campo de piedras amontonadas''. Las memorias de un soldado Jasllego nos cuentan cómo el grupo que se coló por el bosque volvió al dia siguiente con más comida, herramientas y tesoros de los que podían cargar (``Tuvieron que volver al bosque, donde habían abandonado todos los víveres que no podían cargar''). 
\par
Aún con las historias de Asipa y el soldado anónimo, hay preguntas alrededor del saqueo de Detal que permanecen un misterio. Por ejemplo, no se hace mención en ninguna memoria de la época a los cadáveres que los exploradores no pudieron encontrar, ni se explica cómo los exploradores no pudieron reconocer la ciudad (La teoría más aceptada es que confundieron los restos de la ciudad con la cantera de Detal utilizada para obtener la piedra para los edificios y que la frase en cuestión era un comentario sarcástico). Sea como fuere, las noticias del saqueo de Detal corrieron como la pólvora por las filas del ejército Jasllego, y un mes después todas las ciudades fronterizas de Nofarra habían sido saqueadas y destruidas. El único motor del ejército en esos momentos era su propio sustento: Saqueaban una ciudad, y al acabarse la comida se movían a la siguiente. 
\par 
Un general nofarro ideó la estrategia que, al menos durante su vida, le propulsaría a la fama (Su nombre, por desgracia, se perdió en la quema de archivos de Nofarra al final de la guerra, supuestamente para proteger a él y a su familia de represalas por parte del resto de reinos). Tras identificar la siguiente ciudad a la que parte del ejército se movería (Cadel), ordenó evacuar todos los edificios, y los repobló al completo por soldados de la Guardia Real. Tras 3 días fingiendo actividad (Los soldados vestían como campesinos y pretendían trabajar como tales, a pesar de que muchos de ellos nunca habían desempleado ningún otro oficio) las tropas Jasllegas invadieron la ciudad desprotegida (La única ciudad fortificada del reino Nofarro en esta época era la capital), confiados tras media docena de saqueos, y confiando en que la fama que se habían ganado haría las cosas más fáciles. Cuando todos los soldados estaban en casas ajenas comiendo, bebiendo y celebrando, el general anónimo hizo sonar un cuerno, y los soldados Nofarros sacaron sus espadas y atacaron. Según la memoria del soldado anónimo que documentó el asalto a Detal, que fue capturado en Cadel, ``La ciudad se llenó con gritos de horror y espanto, pero esta vez no eran de campesinos y enemigos. Eran de los nuestros. Había tanta sangre en las calles de la ciudad que al menos una docena de los míos, intentando correr hacia un lugar seguro, resbalaron y murieron del golpe. Los pocos supervivientes fuimos capturados y encerrados en una prisión de la que seguramente no pueda salir''. El soldado anónimo moriría en prisión 3 años después de que sus memorias fueran publicadas, según una nota al pie del amigo suyo que consiguió sacar los papeles de prisión durante un intercambio de rehenes.
\par 
El ejército Jasllego, mal comunicado y sin recibir noticias ni órdenes claras, cayó en esta trampa una y otra vez. Los pocos que podían correr morían en bosques y caminos, y los supervivientes eran capturados e incomunicados, incapaces de advertir al resto del ejército del peligro. En tan solo un mes, el ejército Jasllego había sido reducido a un pequeño grupo que sería exterminado por el mismo general en la batalla de Ladca en Enero del 737. 
\par 
Mientras tanto el rey Juasco, tras recibir las buenas noticias de la alianza, comenzó a reunir su ejército. Pero cuando un pequeño grupo de soldados estaba empezando a formarse en la frontera, Nofarra ya estaba preparando las ciudades de esa zona contra la invasión. Tras los horrores ocurridos en los pueblos fronterizos del este, no querían correr ningún riesgo. En Febrero del 737 el rey Nofarro recibía la noticia de la victoria aplastante contra el ejército invasor, y tras otorgar recompensas al General y a su grupo de soldados (Y pedirle que se quedara en esa zona hasta el final del conflicto), ordenó a todas las tropas que se habían reunido a lo largo del reino que acudieran en la defensa del reino en la frontera oeste. 
\par 
Aún sin noticias de la derrota de su aliado, y confiando en que el ejército Nofarro estaría dividido y confuso, Ladislao ordenó a las tropas repartirse por la frontera. Su idea era, según las memorias de su consejero de guerra, capturar todas las ciudades del reino una por una, eliminando la amenaza sinrisista y avanzando hasta la capital del reino. Sin embargo las tropas Nofarras se le adelantaron, y en Marzo del 737, cuando el rey empezaba a recibir noticias de la derrota total de su aliado, una marabunta de soldados nofarros invadió y conquistó la capital del País Juasco. 
\par 
Para mantener la moral de las tropas, que tras la conquista de su capital estaba tan baja que prácticamente se dejaban matar, Ladislao intercambió el territorio conquistado por una gran zona del norte del país. Sin embargo la efectividad de las tropas Juascas se vería seriamente mermada a lo largo de las próximas semanas, y Ladislao fue incapaz de realizar ningún avance más.
\par 
Tras este conflicto inicial, las prioridades de los tres reinos eran bien diferentes. Jarlicia quería reconstruir su ejército, pero debido a la inmensa llamada popular realizada para la anterior maniobra, las aldeas estaban vacías y prácticamente no había voluntarios que reclutar (El efecto de esta tragedia se sigue notando incluso más de 200 años después). Nofarra (O al menos su rey) estaba interesada, simple y llanamente, en destruir a los otros dos reinos, pero la población percibía al propio ejército como tan poderoso que su ayuda era innecesaria, y todos los intentos de reclutar a campesinos para formar una fuerza popular eran recibidos con sublevaciones y linchamientos. El país juasco se estaba empezando a dar cuenta de su posición, y era de su interés el mantener un periodo de tensión entre los reinos. Esto le permitiría evitar un conflicto armado abierto (Sin conocer la negativa de los ciudadanos nofarros a unirse al ejército Ladislao asumía que, con los nuevos territorios adquiridos, un nuevo conflicto podría resultar en la aniquilación del reino), pero al no haber una paz declarada explícitamente, le permitiría minar la autoridad del gobierno Nofarro y atacar a su ejército y su población, siempre que lo hiciera de forma discreta. 
\par 
Este periodo de tensa ``paz'' se prolongaría hasta el tratado de Arla (745), en el que los tres reinos renunciaban a la violencia (Al menos, a la violencia directa). El tratado imponía serias limitaciones al comercio del País Juasco y Jarlicia, sanciones por los ataques a lo largo de años anteriores, multas, y mil inconvenientes más. Sin embargo, con el tratado firmado nadie podía realizar ninguna acción contra el tratado sin declarar de nuevo la guerra.

\section*{La Guerra Templada (745-870}

TO-DO

 y desencadenando una serie de guerras que concluirían con el tratado de Arla en el año 870. Sin embargo, y a pesar de la paz, cada año los líderes de Nofarra se hacían más estrictos y severos, y cada generación de la familia real se tomaba más en serio su misión de conquistar a los demás reinos en pro de la causa sinrisista. Para algunos de ellos, como Alfredo el Serio (910-940), acabar con el humor en el continente sería la misión por la que darían su vida. 

\section*{Postguerra y actualidad (870-)}

TO-DO

Ahora, tras más de 100 años de tensa paz, empiezan a surgir rumores sobre un artefacto histórico que podría alterar el balance de poder. Nadie sabe exactamente de qué se trata, pero todos los reinos están convocando a sus mejores héroes para iniciar una cruzada épica contra sus enemigos, conseguir el artefacto y dominar el continente hasta el fin de los tiempos. A las puertas del nuevo milenio, ni siquiera los reyes saben si serán ellos quienes finalmente conquisten y unifiquen el continente. Ni siquiera si habrá continente que unificar.