\chapter{Apéndice B: Variantes}

Es posible que la gente dispuesta a probar Tres Reinos ya haya jugado al rol antes y sea familiar con las mecánicas y las historias habituales. En este apéndice hay algunas normas, modificaciones y variantes (No exclusivas para este juego) que pueden hacer las partidas bastante más divertidas. No es necesario escoger ninguna de ellas para jugar.

\section{Mecánica general}

\subsection*{Pifias ultracríticas}

Cada vez que un jugador saca un 1 en una tirada de dados, el resultado es absurdamente desastroso.

\subsection*{Por los pelos}

Cuando un jugador obtiene un 1 o un 20, será posible repetir la tirada para salvarse de la pifia (En el caso del 1) o mejorar aún más su resultado (En el caso del 20). Un mal resultado en esta segunda tirada puede anular los buenos resultados de la anterior, o empeorar los malos.

\subsection*{Nadie es perfecto}

Al elegir los valores de los atributos de entre las tiradas de dados, el jugador tiene que añadir la tirada más baja a su selección. 
\par 
Ejemplo: Si los resultados son 14, 5, 10, 8, 12 y 10, el jugador tiene que usar el 5 en uno de los atributos.

\section{Combate}

\subsection*{Carlos Duty}

Un personaje puede curarse de una herida quedándose quieto tres turnos durante un combate sin recibir daño.

\section{Ambientación, Otros}

\subsection*{Replay}

En cualquier momento durante la partida, los jugadores pueden decidir usar un amuleto mágico que les devuelve 5 minutos atrás en el tiempo dentro de la partida. El amuleto acciona una ruleta que elige un evento al azar de una lista.

Se recomienda que la lista contenga principalmente eventos desfavorables para evitar el abuso del botón, y que la lista sólo sea visible por el master. Incluir algún tipo de cooldown nunca está de más (Por ejemplo, una vez cada día transcurrido en el juego).

\subsection*{Me no parlo spraje}

Los diferentes idiomas dentro del juego son representados con diferentes idiomas entre los jugadores. ¡No vale usar un lenguaje común!

\subsection*{Sssshh!!}

Todo lo que los jugadores dicen a lo largo de la partida es repetido por los personajes que manejan, sea cual sea la frase o la situación.

\subsection*{Tabú}

Antes de empezar una partida, los jugadores acuerdan una palabra tabú. Cada vez que un jugador diga esa palabra y el master se de cuenta, la próxima tirada del jugador será un 1 automático

